\chapter{Implementation}

This chapter will present the implementation of the REST algorithm. This will go over the provided details from \cite{zhao2018rest} and the assumptions we made. Without access to the source code some assumptions had to be made. The implementation consists of two parts; reference set construction and trajectory compression.

\section{Reference set construction}
For this part of the algorithm the compression based approach CA-5 was selcted, as it showed the best results. Compression based means that you use the compression ratio as way to measure the current reference sets coverage. If the coverage is too low more trajectories will be added to the set. Trajectories that have sufficient coverage in the set are redundant. The "5" corresponds with the compression ratios threshold for redundancy, meaning trajectories compressed with a CR > 5 are redundant. CA-5 initializes an empty set and iterates over a subset of trajectories. The trajectories from the subset are compressed using the current set, and added to the set if the compression ratio is below 5. As more trajectories are added to the reference set the coverage increases so fewer and fewer will be added. - Hopefully reaching a point where coverage is very high even for a much larger collection of trajectories in the same region.

++ add RTREE
++ discuss the option of adding only the redundant part of trajectory
++ discuss an actual hashset vs vector

When using the compression based approach to build the reference set, there are two possible interpretations of what should be added to the set. The point of the compression based approach is to attempt to compress a trajectory and then based on the compression ratio decide wether it is covered by the set or not. This is defined as a threshold for the compression ratio. In the CA-5 method all trajectories with a compression ratio < 5 are considered not covered. These trajectories are added to the reference set.

When compressing using a reference set you attempt to compress with reference to the trajectories in the set. In most cases, some part of the OT is compressed and some parts are stored directly. However with the strategy mentioned in the previous paragraph the entire trajectory is added to the new reference set, even the parts that were considered redundant. An alternative strategy only adding the non-reduntant parts to the reference set would lead to a smaller reference set while seemingly maintaing the same coverage.


\begin{lstlisting}[
    caption={Psuedo code for reference set construction},
    language=Pascal,
]
Input: T, rs
Output: ReferenceSet
ReferenceSet = []
SampleTrajectorySet = AllTrajectories.take(10%)

for ST in SampleTrajectorySet
    CR = Compress(ST, ReferenceSet)
    if CR < 5 
        ReferenceSet.add(ST)

ReferenceSet
\end{lstlisting}

%Psuedo code for reference set construction

\section{Greedy Spatial Compression}
++ Mention the added option for dtw
Mention lemma 1
Mention the implementation
Mention replacing the conecept of id with referencin in rust - requires all to be in memory, but has benefits


