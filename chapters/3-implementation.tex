\chapter{Implementation}
\label{chap:impl}
In this chapter we will go through how the original version of REST as well as our variants were implemented. Without access to the source code we had to make assumptions for parts of the algorithm, in addition some smaller parts were changed from the original. The code for this algorithm was written in rust and the source code is available on "github.com". The chapter is divided into two sections, detailing the two different parts of the algorithm namely, reference set construction and the compression algorithm.

\section{Reference Set Construction}
\subsection{Original REST implementation}
For the reference set constructino we implement the strategy CA-5, which stands for compression based approach with 5 as a compression ratio thresold. This method uses the attempts uses the compression ratio of of trajectories new to the set to determine wether they are redundant to the set. If a trajectory is compressed with a ratio larger than the threshold it is considered redundant, i.e. the set already covers the trajectories to a sufficient degree. CA-5 begins with an empty reference set and then iterates over a subset of all trajectories. For each iteration it determines wether the trajectory is redundant or not. Non redundant trajectories are added to the reference set, for redundant trajectories nothing is done and the iterations simply continue. See code listing \ref{lst:ca_og} for the pseudo code of reference set construction.

\begin{figure}[t]
    \begin{lstlisting}[
        caption={Psuedo code for reference set construction},
        language=Pascal,
        label=lst:ca_og
    ]
        Input: T, rs
        Output: ReferenceSet
        ReferenceSet = []
        SampleTrajectorySet = AllTrajectories.take(10%)

        for ST in SampleTrajectorySet
            CR = Compress(ST, ReferenceSet)
            if CR < 5 
                ReferenceSet.add(ST)

        ReferenceSet
    \end{lstlisting}
\end{figure}

The reference set grows as trajectories from the subset are added, it will have high coverage for the trajectories in the subset, because each trajectory is either in the set or can be represented by the set. The coverage for all trajectories depends on the size of the input subset. %More on size? And maybe another RQ

The authors of REST used a hashset as the datastructure for the reference trajectories, we chose to use an array instead because the uniqueness of the trajectories in the reference set is not a concern. Trajectories will only be added if they are non-redundant for the current reference set, therefore there is no need to check uniqueness using a hashset. We therefore used an array because it has much cheaper insertions.
\begin{figure}
    \begin{lstlisting}[
        caption={Psuedocode for reference set construction with spatial filter},
        language=Pascal,
        label=lst:ca_sf,
        numbers=left
    ]
        Input: SampleTrajectories, CompressionThreshold
        Output: (ReferenceSet, RTree)
        ----------------------------------------
        ReferenceSet = []
        RTree = new RTree()

        for ST in SampleTrajectorySet
            CompressionRatio = Compress(ST, ReferenceSet)
            if CompressionRatio < CompressionThreshold 
                i = ReferenceSet.len()
                for (pt, j) in ST
                    RTree.insert(pt, (i, j))
                ReferenceSet.push(ST)

        (ReferenceSet, RTree)
    \end{lstlisting}
\end{figure}
One nuance of the algorithm which should be discussed is that the entire trajectory is added to the reference set. A trajectory as a whole can be determined non-redundant, even if some subtrajectories are redundant. The algorithm could then add only the non-redundant parts of the trajectory. This would not be an expensive process as the trajectory has already been compressed and divided into references and points. However, from our interpretation of \cite{zhao2018rest} section 3.3 the trajectory in its entirety is added to the reference set. Therefore we decided to add the the entire trajectory to the set to stay close to the original version of REST in order to answer RQ1 and RQ2.

\subsection{Spatial Filter Variant}

The compression algorithm uses the reference set as a basis for compressing new trajectories. Searching through the entire set for each compression to find candidates to use as references is highly inefficient. Therefore we implement a spatial filter so that the compression algorithm only consideres reference trajectories that are close to the trajectory currently being compressed. The spatial filter is an R-tree containing each point of each trajectory in the reference set. A leaf node in the R-tree contains the position of the point and a tuple index \textit{(i, j)}, \textit{i} is the index of the point's parent trajectory and \textit{j} is the index to the point itself. The psuedo code for reference set construction with a spatial filter is shown in code listing \ref{lst:ca_sf}

\section{Compression Algorithm}
\subsection{Greedy Spatial Compression}
For the compression algorithm we implemented Greedy Spatial Compression \break (GSC) from REST. This is an algorithm that searches greedily for the best result. It consists of two parts: a greedy \acrshort{mrt} search and a wrapper connecting compression sequences of subtrajectories. For a trajectory \textit{T} = [$t_0$, ..., $t_n$] it searches for a matching reference trajectory for the longest possible subtrajectory starting in $t_0$. If no \acrshort{mrt} is found then it adds [$t_0$, $t_1$] uncompressed to the compression sequence. Then it begins a new search starting in $t_1$. If an \acrshort{mrt} $r_{1,m}$ matches [$t_1$, ..., $t_m$], the \acrshort{mrt} is added to the compression sequence. The compression sequence for [$t_0$, ..., $t_m$] would be [$t_0$, $t_1$, $r_{1,m}$]. This process continues until a compression sequence for [$t_0$,, ..., $t_n$] is calculated. The pseudo code for \textit{mrt\_search} is written in algorthm 1 of \textcite{zhao2018rest}. The Rust code of our version \textit{greedy\_mrt\_search} can be found in code listing \ref{lst:ca_expand}. The function's input and output is shown in line 1-6, it has 4 input parameters:

\begin{figure}
    \begin{lstlisting}[
        caption={Greedy MRT Search algorithm from source code written in Rust},
        language=Rust,
        label=lst:ca_expand,
        numbers=left
    ]
    fn greedy_mrt_search<'a>(
        candidate_reference_trajectories: &[&'a [Point]],
        st: &[Point],
        spatial_deviation: f64,
    ) -> Option<(usize, &'a [Point])> {
        let mut length_match_map = HashMap::new();
        for rt in candidate_reference_trajectories {
            let mut current_rt_matches: HashSet<(usize, usize)> = (0..rt.len() - 1)
                .into_iter()
                .filter(|&j|
                    dtw(&st[0..=1], &rt[j..=j + 1]) < spatial_deviation
                )
                .map(|j| (j, j + 1))
                .collect();
    
            let mut matched_st_len = 1;
            while !current_rt_matches.is_empty() {
                matched_st_len += 1;
                current_rt_matches.iter().next().map(|arbitrary_match| {
                    length_match_map
                        .entry(matched_st_len)
                        .or_insert_with(||
                            &rt[arbitrary_match.0..=arbitrary_match.1]
                        )
                });
                
                current_rt_matches = current_rt_matches
                    .iter()
                    .filter(|&(_, rt_end)|
                        (matched_st_len < st.len() - 1) && (*rt_end < rt.len() - 1)
                    )
                    .map(|&(rt_start, rt_end)| {
                        [
                            (rt_start, rt_end),
                            (rt_end, rt_end + 1),
                            (rt_start, rt_end + 1),
                        ]
                        .into_iter()
                        .filter(|&(s, e)| {
                            dtw(&st[..=matched_st_len], &rt[s..=e]) 
                            < spatial_deviation
                        })
                    })
                    .flatten()
                    .collect();
            }
        }
    
        length_match_map.into_iter().max_by_key(|&(k, _)| k)
    }
    \end{lstlisting}
\end{figure}

\begin{itemize}[leftmargin=*]
    \item{\textit{t} - The trajectory being compressed}
    \item{\textit{candidate\_reference\_trajectories} - The reference trajectories used in compression}
    \item{\textit{spatial\_devation} - The spatial devation threshold for MRTs}
    \item{\textit{band} - The band used in the sakoe-chiba DTW}
\end{itemize}

The output is a tuple (\textit{m}, $r_{0,m}$) or None. \textit{m} is the last index of the subtrajectory corresponding to the MRT, the first index is always 0 because of the greedy search strategy. The compressed subtrajectory is given by \textit{t} and \textit{m} as [$t_0$, ..., $t_m$], while $r_{0,m}$ is an MRT for the subtrajectory. The compressed subtrajectory is the longest possible subtrajectory with an MRT from \textit{candidate\_reference\_trajectories}. None is returned if no MRTs were found.

Line 7 in code listing \ref{lst:ca_expand} is the start of the function and intializes a map for subtrajectory - MRT pairs (same structure as the output of the function). Line 8-47 is the code block for search in each reference trajectory \textit{rt}. It initializes \textit{current\_rt\_matches} with all length two MRTs for [$t_0$, $t_1$] (line 9-15). It does this by calculating the dtw distance between [$t_0$, $t_1$] and subtrajectories [$rt_j$, $rt_{j+1}$] for $j = [0, ..., m-2]$ where $m$ is the length of the $rt$. All subtrajectories with dtw distance lower than the $spatial\_deviation$ is considered an MRT. It follows from lemma 1 that this will be the basis for all MRTs. This is because DTW can never decrease the cost to a point later in the matrix.

Next code block is about finding mrts for current\_st+1. The mrts for current\_st are increased like rta rtb and rtab and checked for matches to current\_st+1. This is according to algorithm 1 in rest. An early codeblock writes the result to the a global map. An arbitrary match is selected, because this is bounded lossy compression. THis means its not "best effort", but simply bounded. It could in theory be designed to select the lowest dtw MRT for st05 but from the design of bounded lossy compression any MRT with dtw < spatial threshold is good enough, therefore no resources are spent finding the best ones. In addition there is no guarantee it would be any good. Therefore it is unreliable and therefore not useful. If this is a problem one should change away from bounded lossy compression and not implement at best effort version. How does code block do this? First store arbitrary match, then overwrite itself to the next matches.

In the end the arbitrary mrt for the longest st is returned. Line 50.



In our version of greedy mrt expand it iterates through each \textit{candidate trajectory} and returns the \acrshort{mrt} of the longest subtrajectory. One clarification: the goal is not a long \acrshort{mrt}, but an \acrshort{mrt} for a long ST. By compressing with \acrshort{mrt}s from long STs we need fewer references to compress T in its entirety. The candidate trajectories depend on the spatial filter. Without a spatial filter the candidate trajectories are simply all trajectories in the reference set. With a spatial filter the candidate trajectories are all subtrajectories of reference trajectories that start within a certain spatial filter distance of \textit{t[0]}. This is why each point of the reference trajectory is mapped in the r-tree when the reference set is built. After the longest ST with an \acrshort{mrt} from the set is collected in each iteration of these, the \acrshort{mrt} of the longest ST is again returned (this time global). The code for greedy mrt expand is shown in \ref{lst:ca_expand}. And now the greedy mrt expand finishes. To ensure compression the whole T, a wrapper is placed around greedy mrt expand, which stores the sequence of \acrshort{mrt}s and raw points globally.

We have implemented this conceptually the same way as REST, but with some pruning and practical programming considerations.
For example in our version all searching for one reference trajectory is completed the first time it is loaded to memory. In code listing \ref{lst:ca_expand} line 8 marks the beginning of the block that computes \acrshort{mrt}s for the \textit{current} reference trajectory. The \acrshort{mrt}s computed are added to a global \acrshort{mrt} map.

In addition in the initialization process which checks all length two subtrajectories from T and RT for match is changed. Our version checks all length two subtrajectories from RT with the length two subtrajectory [T[0], T[1]], because this is a greedy algorithm which begins in T[0].



\subsection{DTW}
Each trajectory comparison in the compression has trajectory distance measure as its basis. This was dynamic time warping in \cite{zhao2018rest} and in our implementation. The theory of dtw is covered in 2.4, here we will discuss the implementation of unconstrained dtw aswell as dtw with sakoe-chiba band.

dtw-rs is a rust library with a complete implementation of unconstrained dtw and a partial (not-working) implementation of dtw with the sakoe-chiba band. We made a contribution to this library by completing the sakoe-chiba implementation.