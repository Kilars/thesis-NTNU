\chapter{Implementation}
\label{chap:impl}
This chapter explains the interpretation of REST that is implemented as well as different improvements that were made to speed it up. In this thesis we only look at spatial trajectory compression. In addition, the algorithms we selected from REST are the compression based approach to build reference set and the greedy spatial compression. This is because they showed promising results. Additionally, we explain how the DP-DTW comparison was made. This was coded in Rust and can be found on \href{https://github.com/Kilars/master-code/algo}{https://github.com/Kilars/master-code/algo}.


% In this chapter we will go through how the original version of REST as well as our variants were implemented. Without access to the source code we had to make assumptions for parts of the algorithm, in addition some smaller parts were changed from the original. The code for this algorithm was written in rust and the source code is available on "github.com". The chapter is divided into two sections, detailing the two different parts of the algorithm namely, reference set construction and the compression algorithm.
\section{REST}
In total there are 5 variants implemented. REST, REST\_EXCL, REST-SF, REST-BND, REST-KNN. All detailed in the following sections. The two following sections discuss the interpretation of REST for the compression based approach for reference set construction and the greedy spatial compression for the compression part.
\subsection{Reference Set Construction}
Reference set construction was implemented using the compression-based approach as defined by \textcite{zhao2018rest}, and introduced briefly in section \ref*{sec:REST}. This section will further discuss the reasoning behind the approach, as well as our method of implementation.

An efficient reference set has high \textit{coverage} with low redundancy, \textit{coverage} refers to the extent which the reference set represents the entire set of trajectories. A set with high coverage would yield a high compression ratio for all trajectories in the set. Redundancy refers to how much overlap there is between trajectories in the set. Striking a balance between coverage and redundancy is important, as too high coverage will likely lead to many trajectories in the set, leading to high redundancy. Conversely, tolerating too little redundancy will lead to too few trajectories in the set, resulting in insufficient coverage.

The compression-based approach naturally minimizes redundancy using the threshold and achieves sufficient coverage when using a large sample. It compresses a sample of all trajectories and adds those with compression ratios below a threshold to the set, indicating they are not well represented. The compression ratio threshold affects the redundancy tolerance; a lower threshold indicates a higher tolerance.

The pseudocode for this algorithm can be seen in code listing \ref{lst:ca_og}. The set is initialized as an empty list and trajectories are compressed and added or discarded iteratively. With a sufficient sample size the reference set will have high coverage for the entire set of trajectories.
\begin{figure}[t]
    \begin{lstlisting}[
        caption={Psuedo code for reference set construction},
        language=Pascal,
        label=lst:ca_og
    ]
        Input: T, rs
        Output: ReferenceSet
        ReferenceSet = []
        SampleTrajectorySet = AllTrajectories.take(10%)

        for ST in SampleTrajectorySet
            CR = Compress(ST, ReferenceSet)
            if CR < 5 
                ReferenceSet.add(ST)

        ReferenceSet
    \end{lstlisting}
\end{figure}

\subsection{Compression Algorithm}
For the compression algorithm we implemented Greedy Spatial Compression \break (GSC) from REST. This is an algorithm that searches greedily for the best result. It consists of two parts: a greedy match search and a wrapper connecting compression sequences of subtrajectories.
The matches are called Matchable Reference Trajectories and the definition is taken directly from \textcite{zhao2018rest}.
\begin{quote}
    Definition of Matchable Reference Trajectory (MRT). \textit{Given a sub-trajectory T(i,j) and a spatial deviation threshold $\epsilon_{s}$, its matchable reference trajectory set, denoted as M(T(i,j)), includes all the reference sub-trajectories with less-than-$\epsilon_{s}$ MaxDTW distance with T(i,j), i.e.,}
    \begin{align*}
        \hspace*{-3cm} M(T(i,j)) = \{ & \mathbb{T}(k,g) \mid \mathbb{T} \in \mathcal{R}, i \leq k \leq g \leq \left\lvert \mathbb{T} \right\rvert, \\
        \hspace*{-3cm}                & \text{MaxDTW}(T(i,j), \mathbb{T}(k,g)) \leq \epsilon_{s} \}
    \end{align*}
\end{quote}
From this and the properties of MaxDTW, \textcite{zhao2018rest} also derived a rule.

\begin{quote}
    \label{lemma}
    Lemma 1. \textit{Any sub-trajectory of the MRT of T(i, j) is also an MRT of sub-trajectory of T(i, j).}
\end{quote}

With the definition of an MRT and Lemma 1, the GSC algorithm can be explained.
For a trajectory \textit{T} = [$t_0$, ..., $t_n$] it searches for an MRT for the longest possible subtrajectory starting in $t_0$. If no \acrshort{mrt} is found then it adds [$t_0$, $t_1$] uncompressed to the compression sequence. Then it begins a new search starting in $t_1$. If an \acrshort{mrt} $r_{1,m}$ matches [$t_1$, ..., $t_m$], the \acrshort{mrt} is added to the compression sequence. The compression sequence for [$t_0$, ..., $t_m$] would be [$t_0$, $t_1$, $r_{1,m}$]. This process continues until a compression sequence for [$t_0$, ..., $t_n$] is calculated. The pseudocode for \textit{mrt\_search} is written in algorithm 1 of \textcite{zhao2018rest}. The Rust code of our version \textit{greedy\_mrt\_search} can be found in code listing \ref{lst:ca_expand}. The function's input and output is shown in line 1-6, it has 4 input parameters:

\begin{figure}
    \begin{lstlisting}[
        caption={Greedy MRT Search algorithm from source code written in Rust},
        language=Rust,
        label=lst:ca_expand,
        numbers=left
    ]
    fn greedy_mrt_search<'a>(
        candidate_reference_trajectories: &[&'a [Point]],
        st: &[Point],
        spatial_deviation: f64,
    ) -> Option<(usize, &'a [Point])> {
        let mut length_match_map = HashMap::new();
        for rt in candidate_reference_trajectories {
            let mut current_rt_matches: HashSet<(usize, usize)> = (0..rt.len() - 1)
                .into_iter()
                .filter(|&j|
                    dtw(&st[0..=1], &rt[j..=j + 1]) < spatial_deviation
                )
                .map(|j| (j, j + 1))
                .collect();
    
            let mut matched_st_len = 1;
            while !current_rt_matches.is_empty() {
                matched_st_len += 1;
                current_rt_matches.iter().next().map(|arbitrary_match| {
                    length_match_map
                        .entry(matched_st_len)
                        .or_insert_with(||
                            &rt[arbitrary_match.0..=arbitrary_match.1]
                        )
                });
                
                current_rt_matches = current_rt_matches
                    .iter()
                    .filter(|&(_, rt_end)|
                        (matched_st_len < st.len() - 1) && (*rt_end < rt.len() - 1)
                    )
                    .map(|&(rt_start, rt_end)| {
                        [
                            (rt_start, rt_end),
                            (rt_end, rt_end + 1),
                            (rt_start, rt_end + 1),
                        ]
                        .into_iter()
                        .filter(|&(s, e)| {
                            dtw(&st[..=matched_st_len], &rt[s..=e]) 
                            < spatial_deviation
                        })
                    })
                    .flatten()
                    .collect();
            }
        }
    
        length_match_map.into_iter().max_by_key(|&(k, _)| k)
    }
    \end{lstlisting}
\end{figure}

\begin{itemize}
    \item{\textit{t} - The trajectory being compressed}
    \item{\textit{candidate\_reference\_trajectories} - The reference trajectories used in compression}
    \item{\textit{spatial\_deviation} - The spatial deviation threshold for MRTs}
\end{itemize}

The output is a tuple (\textit{m}, $r_{0,m}$) or None. \textit{m} is the last index of the subtrajectory corresponding to the MRT. $r_{0,m}$ is the MRT itself. The compressed subtrajectory is given by \textit{t} and \textit{m} as [$t_0$, ..., $t_m$], the first index is always 0 because of the greedy search strategy. The compressed subtrajectory is the longest subtrajectory with an MRT from \textit{candidate\_reference\_trajectories}. None is returned if no MRTs were found.

Line 7 in code listing \ref{lst:ca_expand} is the start of the function and initializes a map for subtrajectory - MRT pairs (same structure as the output of the function). Line 8-47 is the code block for search in each reference trajectory \textit{rt}. It initializes \textit{current\_mrts} with all length two MRTs for [$t_0$, ..., $t_i$] where $i = 1$ (line 9-15). It does this by calculating the dtw distance between [$t_0$, $t_1$] and subtrajectories [$rt_j$, $rt_{j+1}$] for $j = 0, 1, ..., m-2$, where $m$ is the length of the $rt$. $current\_mrts$ is the collection of all subtrajectories with dtw distance lower than the $spatial\_deviation$. It follows from \hyperref[lemma]{Lemma 1} that this will be the basis for all MRTs. This is because DTW can never decrease the cost to a point later in the matrix.

The loop in line 17-47 uses $current\_mrts$ to find MRTs for $st_{i+1}$ = [$t_0$, ..., $t_{i+1}$] and stores MRTs to the global set (line 20-26). For each $rt$ = [$rt_0$, ..., $rt_m$] in $current\_mrts$ (matches for $st_i$), three expansions are tested for a match with $st_{i+1}$:
\begin{itemize}
    \item {[$rt_0$, ..., $rt_m$]}
    \item {[$rt_m$, ..., $rt_{m+1}$]}
    \item {[$rt_0$, ..., $rt_{m+1}$]}
\end{itemize}
This is in accordance with the expansion in the REST mrt\_search algorithm. This loop continues, increasing $i$ for each iteration, as long as an MRT can be found for $st_{i+1}$ or until $i+1 = |st|$. In addition, an arbitrary match from the $current\_mrts$ is added to the global map if no entry exists for $st_{i}$, for each iteration. When this loop finishes, the algorithm repeats the process for another reference trajectory. In the end, on line 50 the (ST, MRT) tuple for the longest ST from the global map is returned.

We have implemented this conceptually the same way as REST, but with some pruning and practical programming considerations. For example in our version all searching for one reference trajectory is completed the first time it is loaded to memory. This is opposed to the pseudocode of the REST algorithm which checks for MRT expansions in an arbitrary manner. In addition, the initialization process for MRTs is changed. Our version checks all length two subtrajectories from RT with the length two subtrajectory [$t_0$, $t_1$], because this is a greedy algorithm which begins in T[0]. While the version in REST finds all MRTs for $[t_i, t_{i+1}] \in T$. This difference is due to their version working for both greedy and optimal strategies. This simplification could only be made by specializing the function to only support a greedy search.

Regarding MRT selection when multiple are available, this was done arbitrarily. From the definition, an MRT has a DTW distance to the subtrajectory being compressed lower than the spatial deviation threshold. This means one reference trajectory can have a lower DTW distance than another, while both are considered MRTs. From this one could argue that the algorithm should select the MRT with the lowest distance. However, this implementation of REST applies bounded lossy compression, which means any result within the threshold is valid or "good enough". Therefore, no resources are spent locating the best MRT within the threshold, even though the resources required aren't large because the DTW distance is calculated anyway. If this were implemented, it could be considered "best effort bounded lossy compression". Comparing that to bounded lossy compression would be interesting, but it was not done due to time constrains. It is also likely that it wouldn't make a large difference because a reference trajectory with a significantly lower DTW distance could also match for a longer subtrajectory. This means that if there were multiple MRTs to select, they are likely close in DTW distance to the relevant subtrajectory. The way the reference trajectories expand and how this effects the DTW distance is further discussed in results and discussion. %DISCUSSION REFERENCE
\subsection{Exclusive REST}
One nuance of the reference set construction algorithm which should be discussed is that the entire trajectory is added to the reference set. A trajectory as a whole can be determined non-redundant, even if some subtrajectories are redundant. The algorithm could then add only the non-redundant parts of the trajectory. This would not be an expensive process as the trajectory has already been compressed and divided into references and points. However, from our interpretation of \cite{zhao2018rest} section 3.3 the trajectory in its entirety is added to the reference set. This impacts the reference set by adding more redundancy to the set. Meanwhile, it also leads to more coverage since more of the trajectories are added. Whereas an implementation which only adds the non-redundant parts would have less redundancy and lower coverage. %+ having shorter trajectories since they are "oppstykket".
In order to analyze the difference in these approaches both were implemented. The version adding the entire trajectory to the set is called REST, since this is interpreted as being the original implementation. While the other version only adding non-redundant parts of the trajectory to the set is called Exclusive REST (XREST).
\subsection{Spatial Filter}
% Set construction
The compression algorithm uses the reference set as a basis for compressing new trajectories. Searching through the entire set for each compression to find candidates to use as references is highly inefficient. Therefore we implement a spatial filter so that the compression algorithm only consideres reference trajectories that are close to the trajectory currently being compressed. The spatial filter is an R-tree containing each point of each trajectory in the reference set. A leaf node in the R-tree contains the position of the point and a tuple index \textit{(i, j)}, \textit{i} is the index of the point's parent trajectory and \textit{j} is the index to the point itself. The psuedo code for reference set construction with a spatial filter is shown in code listing \ref{lst:ca_sf}

%Compression, eh kinda same thing
The implementation of the spatial filter influences the input parameter \break \textit{candidate\_reference\_trajectories} for code listing \ref{lst:ca_expand}. With no spatial filter this input is simply all trajectories in the reference set. However, with a spatial filter the rtree selects only the trajectories close to the compressed one. It does this with a range query for latitude and longitude. The point itself is in the center of the range. This can be viewed as creating a box around the first point and selecting all points within the box. The box in our implementation is a square with the length of the sides as an input parameter. A larger square means more points are considered, which might lead to longer runtimes. While a smaller square will have less area coverage, but a faster runtime. More details on balancing coverage and performance can be found in the Results/Discussion section. %Reference results

Note that all points of the reference trajectories are indexed in the R-tree, as shown in \hyperref[lst:ca_sf]{Code Listing \ref{lst:ca_sf}} line 10-12. For reference trajectory $rt = [rt_0, ..., rt_m, ..., rt_n]$, where $rt_m$ is within the range query, $rt_{m,n} = [rt_m, ..., rt_n]$ can be considered as an MRT. As the reference set grows, using the R-tree becomes increasingly faster compared to checking the entire reference set, since more points are pruned.

After the reference set is built from a sample of all trajectories, it should be packed. This is because the data will not change and can be considered static. The Packed Hilbert R-tree would likely perform well in this case since it is simple and efficient with for low dimensionality. Additionally it has a very high storage utilization, which is beneficial if the size of the reference set is a concern. However, due to time constraints, this was not implemented. Nonetheless, the authors recommend exploring this as future work.
\begin{figure}
    \begin{lstlisting}[
        caption={Psuedocode for reference set construction with spatial filter},
        language=Pascal,
        label=lst:ca_sf,
        numbers=left
    ]
        Input: SampleTrajectories, CompressionThreshold
        Output: (ReferenceSet, RTree)
        ----------------------------------------
        ReferenceSet = []
        RTree = new RTree()

        for ST in SampleTrajectorySet
            CompressionRatio = Compress(ST, ReferenceSet)
            if CompressionRatio < CompressionThreshold 
                i = ReferenceSet.len()
                for (pt, j) in ST
                    RTree.insert(pt, (i, j))
                ReferenceSet.push(ST)

        (ReferenceSet, RTree)
    \end{lstlisting}
\end{figure}
\subsection{KNN}
Ye I did it
\section{MaxDTW}
MaxDTW has been implemented locally in Rust following the definitions in \ref{sec:dtw}. One important efficiency gain was through memoization of DTW solutions. \\ \textit{greedy\_mrt\_expand} in code listing \ref{lst:ca_expand}, expands with one point at a time, this is like adding one row and column to the DTW matrix. Without memoization, the entire matrix must be recalculated. However, with memoization, only the new column and row need to be calculated, using the previously computed accumulated cost matrix. Additionally an optional Sakoe-Chiba band was implemented like described in \ref{sec:sakoe}.

At one point, we considered using the external Rust library \textit{dtw\_rs} instead of a custom implementation. To prepare for this, we contributed to it by finishing the Sakoe-Chiba band implementation. However, due to the aforementioned memoization, which requires some state awareness from the local environment, integrating this library demanded too many modifications. Consequently, we decided to develop our own implementation. Although this contribution was not used in our final results, it is available here: \\ \href{https://github.com/shshemi/dtw-rs/pull/1/files}{https://github.com/shshemi/dtw-rs/pull/1/files}.

\section{Douglas Peucker DTW}
\section{Data Set}