This section discusses the dataset used for the experiments and the preprocessing steps. The dataset consists of taxi trips that occurred in the city of Porto over the course of a year, from \textcite{porto}. Each taxi trip is represented as a separate trajectory in the dataset. Trajectories are made up of polylines of points with latitudes and longitudes, along with some metadata. The points were logged at 15-second intervals using mobile terminals in the taxis. In total, there are 1.7 million trajectories in the dataset. The separation of trips into individual trajectories is a key feature of this dataset. The Porto dataset only records movements when the taxi is active, traveling from point A to point B, unlike other datasets that include location data for the entire day. The type of trajectories generated from a trip is more interesting than the downtime between trips. During downtime, the taxis typically hover in frequently traveled areas or drive to the next pickup location. These types of trajectories are less interesting because they do not always represent the fastest path and are more challenging to interpret.

To prepare the data for spatial compression, the polylines were written to a CSV file, and the metadata was discarded. Additionally, any polylines with a length of less than two points were removed, as this data is invalid, and the compression algorithms require at least two points in each trajectory. Through analysis, it was discovered that some trajectories had large jumps in location, indicating invalid data. Although these trajectories do not resemble realistic taxi movement, more time was not spent on data cleaning as the purpose of this thesis is to compare different compression algorithms. It is acceptable for some trajectories to be unrealistic as long as the algorithms are being tested on the same data. However, the potential impact of unrealistic trajectories should be considered if it is reasonable to believe that they may affect the algorithms differently.