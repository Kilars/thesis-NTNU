\chapter*{Sammendrag}

Geodata gir verdifull innsikt i bevegelsesmønstre for kjøretøy og mennesker. Ettersom mengden av denne typen data fortsetter å vokse, oppstår det utfordringer knyttet til lagring og overføring. En vanlig løsning er datakomprimering. I denne avhandlingen analyserer vi en ny referansebasert og datadreven komprimeringsalgoritme spesifikk for stidata, kalt \textit{REST}. Algoritmen utnytter datasettets karakteristikker for å effektivt komprimere hver enkelt sti. Resultatene er lovende, men det å bruke "Dynamic Time Warping" til stisammenligning er ressurskrevende. Derfor har vi utviklet en versjon som forbedrer kjøretiden. Denne oppnår en komprimeringsgrad på fem, samtidig som den er ni ganger raskere enn den opprinnelige algoritmen. Videre indikerer resultatene at REST, sammenlignet med andre ikke-referansebaserte metoder gir bedre resultater både med tanke på kjøretid og komp-rimeringsgrad.
