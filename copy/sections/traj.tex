\section{Trajectory Compression}
\label{sub:traj}
\subsection{Methods in Trajectory Compression}
A trajectory is a sequence of geospatial points that represents a path. Optionally, a timestamp of the point registration can also be stored. The goal of compression is to reduce the size of the trajectory by storing in a compressed form. This is measured by the compression ratio, which is the ratio between the uncompressed and compressed data. Trajectory compression can be classified into three main categories: streamed/batched, lossless/lossy, and spatial-only/spatio-temporal.

The primary distinction between streamed and batched compression lies in the amount of data processed at a time. Batched compression is generally simpler because it allows access to the entire trajectory at once. In contrast, streamed compression can only handle one segment at a time due to memory limitations, which prevents loading the entire trajectory. This limitation complicates the process as the complete scope of the trajectory remains unknown. A common approach for managing streamed data is the use of a Sliding Window, which forms the basis of many algorithms. Batched compression, on the other hand, requires more storage and concentrated processing power. It necessitates loading the entire trajectory onto disk and processing it in a single operation. Due to its additional resource requirements and knowledge of the entire trajectory, batched compression outperforms streamed compression in terms of compression ratio and accuracy. Nonetheless, streamed compression can be advantageous in certain situations. Its ability to compress data as it is received, eliminating the need for intermediate storage, can outweigh the superior compression achieved by batched methods in terms of practicality. In addition it's absolutely necessary for real time systems.% Build on real time systems

Lossy algorithms sacrifice accuracy for more efficient storage. Lossless algorithms reduce the size without losing any information. They generally have worse performance than lossy algorithms as a result of this. Within lossy compression, there is a trade-off of "how lossy." This means that you can balance the fidelity of the compression with the compression ratio. Greater fidelity results in lower compression ratio, and vice versa. This is called "bounded lossy compression". Modern algorithms, such as REST, perform bounded lossy compression, rather than a binary choice between lossy or lossless compression.

The distinction between spatial-only and spatio-temporal lies in the consideration of the temporal aspect, that is, whether time is taken into account or not. Spatial-only compression focuses solely on the spatial differences between trajectories. \cite{SpatiotemporalComp}, define two spatiotemporal concepts that can improve trajectory compression: \textit{time-ratio distance} and \textit{speed difference threshold}.

The time-ratio distance represents the distance between two synchronized points: one point from the original trajectory and another point from the compressed trajectory. The traditional error metric used in compression is the perpendicular distance between a point and the new trajectory. However when using synchronized points the distance will change based on time. This was first defined by \cite{SpatiotemporalComp} and has now been formalized under the term Synchronized Euclidean Distance (SED). Figure \ref{fig:sed} illustrates the concept of time-ratio distance, which will be further discussed in section \ref{subsub:SED}

The speed difference threshold is the concept of comparing speeds between sequential segments of the trajectory. The speed is not expected to be directly in the data; rather it iss calculated as $ \Delta d / \Delta t $, where \textit{d} is distance and \textit{t} is time. When the speed difference is large this indicates a sudden movement or turn. These points are considered important for the internal shape of the trajectory. Therefore, ensuring no points with a speed difference larger than a certain threshold are removed during compression can increase the fidelity of the trajectory when considering the temporal aspect.

The results in the experiments by \cite{SpatiotemporalComp} show that implementing time-ratio distance and a speed difference threshold had a slight improvement in compression ratio and a significant reduction of error. Although the error metric was spatiotemporal and the algorithms used in comparison were not, therefore the results are not that surprising. Meanwhile, \cite{Sun2016} claims that spatiotemporal is simple and efficient with the ability to maintain internal features in trajectories. However, they are unpopular because existing algorithms only consider speed which may lead to greater errors and break the holistic geometrical characteristics of trajectories.
